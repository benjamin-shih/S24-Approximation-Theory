\section{Week 1: 2/22/2024}\sectionmark{Week 1}
Chebyshev interpolants have a very strong approximation properties, as opposed to uniformly spaced points. They are the points that correspond to the real part of equispaced points on the unit circle in the complex plane. That is, the Chebyshev points are 
\[ x_j = \cos\left(\frac{j\pi}{n}\right) \]
and a key property is that they ``collect'' near the ends of the interval in higher density. Namely, this key property is that each point is, on average, the same distance away from every other point.
For the most part, we deal with approximating functions on the interval $[-1, 1]$, which any function on any interval $[a, b]$ can be scaled to. 

There are connections that can be drawn between the Chebyshev, Fourier, and Laurent settings, with each being used in numerical, complex, and real analysis heavily, respectively. In the Chebyshev settings, we approximate functions $f(x), x \in [-1, 1]$ with the form
\[ f(x) \approx \sum_{k=0}^n a_kT_k(x) \]
while using $z \in S^1 \subset \mathbb{C}$ equispaced points on the complex plane gives us the Laurent setting with Laurent polynomials
\[ F(z) = F(z^{-1}) = \frac{1}{2}\sum_{k=0}^n a_k (z^k + z^{-k}) \]
and finally using the angle $\theta \in [-\pi, \pi]$ to define $\mathcal{F}(\theta) = F(e^{i\theta}) = f(\cos(\theta))$ gives us Fourier series as
\[ \mathcal{F}(\theta) \approx \frac{1}{2}\sum_{k=0}^n a_k(e^{ik\theta} + e ^{-ik\theta}) \]
Their corresponding canonical grid systems are as follow:
\begin{center}
    \begin{tabular}{cc}
        Chebyshev points & $x_j = \cos\left(\frac{j\pi}{n}\right), \quad 0 \leq j \leq n$ \\
        Roots of unity (Laurent) & $z_j = e^{\frac{j\pi}{n}}, \quad -n+1\leq j \leq n$ \\
        Equispaced points (Fourier) & $\theta_j = \frac{j\pi}{n}, \quad -n+1 \leq j \leq n$
    \end{tabular}
\end{center}

\subsection{Chebyshev Series and Polynomials}
\begin{dfn}[$k$-th Chebyshev polynomial]
    The $k$-th Chebyshev polynomial is the real part of the function $z^k$ on the unit circle; i.e.
    \[ T_k(x) = \Re (z^k) = \frac{1}{2}(z^k + z^{-k}) = \cos(k\theta) \]
\end{dfn}

\begin{theorem}[Existence of Chebyshev Series] Suppose that $f$ is Lipschitz on $[-1,1]$, i.e. that there exists $C \in \R$ such that $|f(x) - f(y)| \leq C|x - y|$ for any $x, y \in \R$. Then $f$ admits a unique representation as a Chebyshev series
\[ f(x) = \sum_{k=0}^\infty a_k T_k(x) \]
which is absolutely and uniformly convergent. The coefficients $a_k$ are given by
\[ a_k = \frac{2}{\pi} \int_{-1}^1 \frac{f(x)T_k(x)}{\sqrt{1-x^2}} dx \]
for $k \geq 1$, and for $k=0$ by the same formula with a $\frac{1}{\pi}$ factor instead.
\end{theorem}